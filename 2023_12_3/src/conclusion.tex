\documentclass[UTF8]{ctexart}
\usepackage{graphicx}
\usepackage{amssymb}
\usepackage{subfiles}
\usepackage{amsmath}
\usepackage[margin=1in]{geometry}
\begin{document}
\section{conclusions}
\paragraph{\quad}In this work, the multiphase Riemann-SPH method is adopted to 
        study the air-cushion effect and slamming load in water entry. Through 
        the test of slamming of a flat plate, the accuracy and convergence of 
        the adopted method are validated, and then the influences of the air cushion 
        and the plate length on the slamming load are discussed. Results show that for 
        the water entry problem involving the air cushion, the buffering effect of the 
        air is vital for the accurate prediction of the slamming load, and ignoring the 
        air will largely overestimate the peak of the slamming load. For the water slamming 
        of the plate, with the increase of plate length, the contact area between the plate 
        and the water increases, which results in the larger slamming load of the plate but 
        more obvious buffering effect of the air cushion. That is, for the longer plate, the 
        slamming force peak of the plate will decrease in a larger proportion due to the existence of air.
\paragraph{\quad}本文中,采用多相黎曼-SPH方法研究入水过程中的气垫效应和砰击载荷。通过平板砰击实验,验证了
                所采用方法的精确性和收敛性,随后讨论了气垫和板长对于砰击载荷的影响。结果表明包含气垫的
                入水问题,空气的缓冲效应对于砰击载荷的精确预测至关重要,忽略空气将极大地过高估计砰击载荷的
                峰值。对于平板与水的砰击,随着板长的增加,板和水间的接触面积增大,会导致更大的板的砰击载荷,
                但更明显的气垫缓冲效应。即板长越长,对于越长的板,由于空气的存在,其砰击载荷的峰值将会减小。

\paragraph{\quad}Besides, focusing on the air-cushion effect in the water entry, the slamming of a LNG tank 
                insulation panel is simulated. It is demonstrated that the air-cushion effect has a great 
                influence on the time-space dis-tributions of the slamming loads. The peak of the slamming 
                pressure increases with the increase of the impact velocity, while the change of the impact 
                velocity has little influence on the formation position of air cushion when the impact velocity 
                ranges from 2.5m/s to 5.5m/s. For the study on the panels with different deadrise angles, it is 
                found that generally, for the panel with a smaller deadrise angle, the buffering effect of the 
                air cushion is more significant and thus leading to lower slamming pressure peak. Furthermore, 
                the change of the deadrise angle of the structure will influence the position of the air bubble 
                generated under the panel and further affect the local slamming loads of the structure.
\paragraph{\quad}此外,本文模拟了LNG船绝缘板的砰击,重点研究了入水过程中的气垫效应。研究证实了气垫效应对砰击载荷的时空分布有
                极大的影响。砰击压力的峰值随着砰击速度的增大而增大,然而当砰击速度在2.5m/s到5.5m/s间,砰击速度的变化对气垫的
                生成位置影响较小。对于研究不同静升角下的面板,可以看出一般而言,对于越小的静升角下的板,气垫的缓冲效应更显著,因此
                导致了更低的砰击压力峰值。此外,静升角的变化将会影响到板底气泡的生成,并经一步影响结构物的砰击载荷。
\end{document}