\documentclass[UTF8]{ctexart}
\usepackage{graphicx}
\usepackage{amssymb}
\usepackage{subfiles}
\usepackage{amsmath}
\usepackage[margin=1in]{geometry}
\begin{document}
\section{abstract}
\paragraph{\quad}The impact of structure on water is an important and practical issue 
                in ocean engineering. For some cases, the influence of air on the impact 
                characteristics is non-negligible. In particular, when the flat-bottomed 
                structure impacts on water such as the emergency landing on the water of 
                an aircraft and helicopter, the air cushion will be formed which buffers 
                the impact of the structure, thereby reducing its slamming load. In this 
                paper, considering the advantages of the Riemann solver in dealing with 
                discontinuities, a multiphase Riemann-SPH method using the PVRS Riemann 
                solver is applied to analyze the air-cushion effect and slamming load in 
                water entry problems. To reduce the numerical dissipation led by the 
                Riemann solver, a dissipation limiter for the PVRS Riemann solver is given.
                Through the test of the water slamming of the plate, the accuracy and convergence 
                of the adopted method are firstly validated. Then, the influences of the air cushion 
                and the plate length on the slamming load are discussed. Finally, a complex 
                engineering problem, i.e., the slamming of the LNG tank insulation panel is 
                simulated, and the influences of the impact velocity and deadrise angle on slamming 
                load characteristics are analyzed.
\paragraph{\quad}在海洋工程中,结构物对水的冲击是重要且实际的问题。在某些情况下,空气流动对砰击特性的影响是不可忽略的。
                特别是平底结构物对水的砰击,例如飞机和直升机的水面迫降时,会形成缓解结构物冲击的气垫,从而减小砰击载荷。
                本文基于黎曼求解器善于处理非连续性问题的优势,应用一种使用PVRS黎曼求解器的多相黎曼-SPH方法分析研究了
                入水问题中的气垫效应和砰击载荷。为了减少由黎曼求解器引起的数值耗散,给出了PVRS黎曼求解器的耗散限制器。
                通过板的水击试验,首次验证了所采用方法的准确性和收敛性。随后讨论了气垫和板长对砰击载荷的影响。最后模拟了
                一个复杂的工程问题,即LNG船保温板的砰击模拟,并研究了砰击速度和静升角对砰击载荷特性的影响。
\end{document}