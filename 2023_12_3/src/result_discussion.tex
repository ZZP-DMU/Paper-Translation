\documentclass[UTF8]{ctexart}
\usepackage{graphicx}
\usepackage{amssymb}
\usepackage{subfiles}
\usepackage{amsmath}
\usepackage[margin=1in]{geometry}
\begin{document}
\section{Numerical result and discussion-数值结果和讨论}
\paragraph{\quad}In this section, firstly, the accuracy and convergence 
            of the adopted multiphase Riemann-SPH method are validated 
            through the test of the slamming of a flat plate. Then, 
            the influences of the air cushion and the plate length on the 
            slamming load are discussed. Finally, focusing on the air-cushion 
            effect in the water entry, a complex engineering problem, i. e., 
            the slamming of the LNG tank insulation panel is simulated, and the 
            influences of the impact velocity and deadrise angle on slamming load 
            characteristics are analyzed.
\paragraph{\quad}本节通过平板砰击实验验证了采用的黎曼-SPH方法的精确性和收敛性。随后,讨论了
                气垫效应和板长对砰击载荷的影响。最后,重点研究了入水过程中的气垫效应,
                模拟了一个复杂的工程问题,即LNG船保温板的砰击,并分析了砰击速度和静升角
                对砰击载荷特性的影响。

\subsection{slamming of flat plate-平板砰击}
\subsubsection{Validation-验证}
\paragraph{\quad}This subsection aims to verify the accuracy and convergence 
                of the adopted SPH method by the test of water slamming of 
                flat plate which is carried out by Ma et al. (2016). The sketch 
                of the numerical model is shown in Fig. 1, in which$ M$,$ l $and $U_{impact}$ 
                stand for the mass, the length and the impact velocity of the plate, 
                respectively.$ H $and$ W $denote the depth and width of water domain, 
                respectively. In the experiment, the plate has the length of$ l = 0.25 m$ 
                and the mass of $M = 32 kg$. This plate is released at rest initially and 
                falls freely, and it impacts the water at the velocity of $5.5 m/s$. In our 
                simulation, the width and depth of the water domain are set to $W = 1.2 m$ 
                and $H = 0.4 m$, respectively. In addition, the reference density of water 
                and air is set to $1000kg/m^3$ and$1kg/m^3$, respectively, and the particle 
                spacing $\Delta x$ is set to 0.002 m. In literature (Ma et al., 2016), the moment 
                when the plate impacts the water is defined as the origin of the time coordinate. 
                For easy to compare, the time coordinate of the numerical results is translated 
                to make it consistent with that of the literature (Ma et al., 2016).
\paragraph{\quad}这一小节旨在通过由Ma et al.(2016)提出的平板砰击实验验证所采用的SPH方法的精确性和收敛性。
                数值方法的示意图如Fig.1所示,其中$M$,$l$,$U_{impact}$分别表示板的质量,长度和砰击速度。
                $H$和$W$分别表示水域的深度和长度。在实验中,板长$ l = 0.25 m$,板的质量为 $M = 32 kg$.
                板由静止释放并自由下落,最后以$5.5m/s$的速度与水砰击。在本文的数值模拟中,水域的宽度和深度
                分别设置为$W=1.2m$和$H=0.4m$.此外,水和空气的参考密度分别设置为$1000kg/m^3$和$1kg/m^3$,
                粒子的间距$\Delta x$设置为$0.002m$.在文献(Ma et al.,2016),平板刚与水砰击的时刻定义为
                时间坐标的原点。为了便于对比,对本文数值结果的时间坐标平移,使之与文献(Ma et al.,2016)一致。

\paragraph{\quad}From the velocity field shown in Fig. 2, it can be observed that clear 
                circulating air jet surrounding the plate is generated during the falling process. 
                This is because the air under the plate is compressed during the falling of the 
                plate, causing the pressure of the air under the plate to increase and the pressure 
                of the air above the plate to decrease. As a result, the air flows around the edges 
                of the plate, which in turn forms a circulating air vortex as displayed in Fig. 2. 
                When the plate is about to contact with the water, the velocity of the air escaping 
                from both sides is very large even over $100 m/s$, although the impact velocity of the 
                plate is only$ 5.5 m/s $in this case. Similar situation can be found in the research of 
                Lind et al. (2015) in which it is believed that the high-speed subsonic flow is readily 
                formed for the air in the water slamming of plate even in a moderate impact event. 
                During the impact of plate, a part of air is trapped under the plate and forms the air 
                cushion for the slam plate, as shown in Fig. 3. At the same time, the pressure distribution 
                of the slam plate is displayed. Aiming at the numerical prediction of the slamming load 
                of the plate, the time histories of the pressure of the plate center and the slamming force 
                of the whole plate are given in Fig. 4 and Fig. 5, respectively. It can be found that 
                generally, the time history of the pressure at the center of plate obtained by adopted 
                the present SPH method agrees well with the FVM result provided by Ma et al. (2016) but a 
                little lower than the experimental data in (Ma et al., 2016). For the slamming force of 
                the whole plate, the present result obtains a higher peak compared with the SPH result of 
                Yang et al. (2020), and the result obtained by present SPH method shows good agreement 
                with the results of experiment and FVM method in Ma et al. (2016). In general, for the 
                present test of the water slamming of flat plate involving air-cushion effect, the present 
                method has good accuracy in predicting the slamming load of structure.
\paragraph{\quad}从速度场如Fig. 2,可以明显看出在下落过程中板周围产生了环状空气射流。这是因为板下空气在下落过程中被压缩,导致板下
            空气压力升高而板上空气压力下降。因此,空气绕过平板边缘流动,进而形成如Fig. 2所示的环状空气漩涡。尽管此时板的砰击速度只有$5.5m/s$
            当平板即将接触水面
            时,从板端逸出的空气速度非常大,甚至超过$100m/s$.类似的情况可以在Lind et al. (2015)的研究中发现,他认为对于空气而言在板与水的砰击过程中,
            即使在缓和的砰击情况下高速亚音速流也会快速形成。在板的砰击过程中,一部分空气困在板下并为砰击板形成缓冲,如Fig. 3所示。
            同时压力分布也如图所示。针对板的砰击载荷的预测,板中心压力的时间历程和整个板的砰击力分别由Fig. 4和Fig. 5给出。
            结果表明,由本文采用的SPH方法获得的板中心压力时间历程和由Ma et al. (2016)获得的FVM结果吻合的很好,但略低于
            (Ma et al.,2016)的实验数据。至于整个板的砰击力,本文的结果比Yang et al. (2020)的SPH结果的峰值更高,但
            本文SPH方法获得的结果和实验结果以及Ma et al.(2016)有很好的一致性。总而言之,对于包含气垫效应的平板和水的砰击实验,
            当前本文的方法表现出良好的结构物砰击载荷的预测精度。

\paragraph{\quad}Furthermore, also through this numerical test, the convergence property of the 
                adopted SPH model is checked. Here, the simulations with the particle resolutions 
                of Δx = 0.003m and Δx = 0.0015m are performed, respectively. Combined with the 
                numerical results with Δx = 0.002m, the time histories of the pressure at the plate 
                center and the slamming force of whole plate with three different particle resolutions 
                are displayed in Fig. 6. It can be found that similar results of the pressure at the 
                plate center and the slamming load of plate are obtained with the particle resolution 
                increasing from Δx = 0.002m to Δx = 0.0015m, which indicates that the adopted SPH 
                scheme has a good convergence.
\paragraph{\quad}此外,此次数值试验验证了采用的SPH方法的收敛性质。这里分别进行了粒子分辨率为$\Delta x = 0.003m$
                和$\Delta x = 0.0015m$的模拟。结合$\Delta x = 0.002 m$的数值结果,不同粒子分辨率的板中心压力
                的时间历程和整个板的砰击压力如Fig. 6所示。可以发现粒子分辨率从$\Delta x = 0.002m$ 增加的$\Delta x = 0.0015m$
                时,板中心压力的时间历程和整个板的砰击力有类似的数值结果,这表明本文采用的SPH方法有良好的收敛性。

\end{document}